



\newpage
\section{Easy Problems (1-12)}

\begin{enumerate}
    \item Simplify: $\log_3(9)$
    
    \item Simplify: $\log_2(8)$
    
    \item Simplify: $\log_{10}(100)$
    
    \item Evaluate: $\log_5(1)$
    
    
    \item Evaluate: $\log_7(7)$
    
    \item Simplify: $\log_2(4) + \log_2(8)$
    
    \item Simplify: $\log_3(27) - \log_3(3)$
    
    
    \item Simplify: $2\log_5(5)$
    
    \item Evaluate: $\log_{10}(1000)$
    
\end{enumerate}

\section{Medium Easy Problems (13-24)}

\begin{enumerate}[resume]
    \item Express as a single logarithm: $\log_3(5) + \log_3(7)$
    
    \item Express as a single logarithm: $\log_2(12) - \log_2(3)$
    
    \item Simplify: $\log_4(x^3)$
    
    \item Express as a single logarithm: $2\log_5(3)$
    
    \item Simplify: $\log_7(49x)$
    
    \item Express as a single logarithm: $\log_6(8) + \log_6(2) - \log_6(4)$
    
    \item Expand: $\log_2(5xy)$
    
    \item Expand: $\log_3\left(\frac{x}{y}\right)$
    
    \item Simplify: $\log_5(25) + \log_5(5)$
    
    \item Express as a single logarithm: $3\log_4(2) + \log_4(5)$
    
    \item Expand: $\log_{10}(x^2y^3)$
    
    \item Simplify: $\log_8(64) - 2\log_8(2)$
\end{enumerate}

\section{Medium Problems (25-36)}

\begin{enumerate}[resume]
    \item Expand completely: $\log_2\left(\frac{x^3y^2}{z^4}\right)$
    
    \item Express as a single logarithm: $\frac{1}{2}\log_5(x) - 3\log_5(y) + \log_5(z)$
    
    \item Solve for $x$: $\log_3(x) + \log_3(x+2) = \log_3(15)$
    
    \item Expand: $\log_7\left(\frac{\sqrt{x}}{y^2z}\right)$
    
    \item Express as a single logarithm: $2\log_4(x) + \frac{1}{3}\log_4(y) - 4\log_4(z)$
    
    \item Solve for $x$: $\log_2(x-1) - \log_2(x+1) = \log_2(3)$
    
    \item Change of base: Express $\log_3(7)$ in terms of common logarithms
    
    \item Change of base: Express $\log_5(12)$ in terms of natural logarithms
    
    \item Simplify: $\log_9(27) + \log_9(3)$
    
    \item Expand: $\log_{10}\left(\frac{x^2\sqrt{y}}{z^3}\right)$
    
    \item Express as a single logarithm: $\log_6(x) + 2\log_6(y) - \frac{1}{2}\log_6(z)$
    
    \item Solve for $x$: $2\log_4(x) = \log_4(9)$
\end{enumerate}

\section{Medium Hard Problems (37-46)}

\begin{enumerate}[resume]
    \item Expand completely: $\log_3\left(\frac{x^2\sqrt[3]{y}}{(z+1)^4}\right)$
    
    \item Express as a single logarithm: $\frac{2}{3}\log_7(x) - \frac{1}{4}\log_7(y) + \frac{3}{2}\log_7(z)$
    
    \item Solve for $x$: $\log_5(x+3) + \log_5(x-3) = \log_5(16)$
    
    \item Change of base: Simplify $\frac{\log_2(8)}{\log_2(16)}$
    
    \item Expand: $\log_4\left(\frac{x^3\sqrt[4]{y^3}}{(z^2+1)^{3/2}}\right)$
    
    \item Solve for $x$: $\log_3(x) + \log_3(x-6) = \log_3(7)$
    
    \item Express as a single logarithm and simplify: $\log_8(4) + \log_8(16) - \log_8(2)$
    
    \item Change of base: Show that $\log_a(b) \cdot \log_b(c) = \log_a(c)$
    
    \item Solve for $x$: $\log_2(x-1) + \log_2(x+3) = \log_2(2x+6)$
    
    \item Expand: $\log_{10}\left(\frac{(x+1)^2\sqrt[3]{y-2}}{(z^2+4)^{5/3}}\right)$
\end{enumerate}

\section{Hard Problems (47-52)}

\begin{enumerate}[resume]
    \item Express as a single logarithm and simplify: 
    $\frac{1}{3}\log_6(216) + \frac{2}{5}\log_6(32) - \frac{3}{4}\log_6(81)$
    
    \item Solve the system:
    \begin{align}
        \log_2(x) + \log_2(y) &= 5 \\
        \log_2(x) - \log_2(y) &= 1
    \end{align}
    
    \item Change of base: Prove that $\log_a(b) = \frac{1}{\log_b(a)}$ for $a,b > 0$, $a,b \neq 1$
    
    \item Expand completely: $\log_5\left(\frac{x^{2/3}(y+1)^{3/4}\sqrt[5]{z-2}}{(w^2-1)^{4/5}(t+3)^{1/6}}\right)$
    
    \item Solve for $x$: $\log_4(x+1) - \log_4(x-1) = \log_4\left(\frac{x+1}{x-1}\right) - \log_4(3)$
    
    \item Change of base challenge: If $\log_2(3) = a$ and $\log_3(5) = b$, express $\log_2(45)$ in terms of $a$ and $b$.
\end{enumerate}

\section{Extra Hard Problems (53-72)}

\begin{enumerate}[resume]
    \item Solve for $x$: $\log_2(\log_3(\log_4(x))) = 0$
    
    \item Express as a single logarithm: $\log_a(b) + \log_b(c) + \log_c(a)$ where $a$, $b$, $c > 0$ and $a$, $b$, $c \neq 1$
    
    \item Solve the equation: $\log_x(8) + \log_{x^2}(8) = 3$
    
    \item If $\log_2(x) + \log_4(x) + \log_8(x) = 11$, find $x$.
    
    \item Prove that for $a > 0$, $a \neq 1$: $\log_a(x) + \log_{a^2}(x) + \log_{a^3}(x) = \log_a(x) \cdot \left(1 + \frac{1}{2} + \frac{1}{3}\right)$
    
    \item Solve for $x$: $\log_{\sqrt{x}}(16) = 4$
    
    \item Find all solutions: $\log_3(x^2 - 6x + 10) = \log_3(2)$
    
    \item Express in simplest form: $\frac{\log_a(b)}{\log_{a^2}(b^3)} \cdot \frac{\log_{a^3}(b^2)}{\log_a(b^4)}$
    
    \item Solve the system:
    \begin{align}
        \log_2(xy) &= 6 \\
        \log_2(x^2y) &= 9
    \end{align}
    
    \item If $a^{\log_b(c)} = c^{\log_b(a)}$, prove this identity and find when it holds.
    
    \item Solve for $x$: $2^{\log_2(x)} + 3^{\log_3(x)} = x + 1$
    
    \item Find the value of: $\log_2(\log_2(\log_2(256)))$
    
    \item Solve: $\log_x(x+6) = \log_{x+6}(x) + \frac{1}{2}$
    
    \item Express as a single logarithm: $\sum_{k=1}^{n} \log_a(k)$
    
    \item Solve for $x$: $\log_{x-1}(x+1) + \log_{x+1}(x-1) = \frac{5}{2}$
    
    \item If $\log_a(x) = p$ and $\log_a(y) = q$, express $\log_{a^2}(\sqrt{x^3y^5})$ in terms of $p$ and $q$.
    
    \item Find all real solutions: $(\log_2(x))^2 - 5\log_2(x) + 6 = 0$
    
    \item Prove that: $\log_a(b) \cdot \log_b(c) \cdot \log_c(d) \cdot \log_d(a) = 1$
    
    \item Solve the functional equation: $f(x) + f(1-x) = 1$ where $f(x) = \log_2\left(\frac{2^x}{1+2^x}\right)$
    
    \item Challenge: If $\log_2(a) + \log_3(b) + \log_4(c) = \log_6(abc)$, find the relationship between $a$, $b$, and $c$.
\end{enumerate}

\newpage

\section*{Answer Key}

\subsection*{Easy Problems (1-12)}
\begin{enumerate}
    \item $2$
    \item $3$
    \item $2$
    \item $0$
    \item $1$
    \item $5$
    \item $2$
    \item $2$
    \item $3$
\end{enumerate}

\subsection*{Medium Easy Problems (13-24)}
\begin{enumerate}[start=13]
    \item $\log_3(35)$
    \item $\log_2(4) = 2$
    \item $3\log_4(x)$
    \item $\log_5(9)$
    \item $2 + \log_7(x)$
    \item $\log_6(4)$
    \item $\log_2(5) + \log_2(x) + \log_2(y)$
    \item $\log_3(x) - \log_3(y)$
    \item $3$
    \item $\log_4(40)$
    \item $2\log_{10}(x) + 3\log_{10}(y)$
    \item $2$
\end{enumerate}

\subsection*{Medium Problems (25-36)}
\begin{enumerate}[start=25]
    \item $3\log_2(x) + 2\log_2(y) - 4\log_2(z)$
    \item $\log_5\left(\frac{\sqrt{x}z}{y^3}\right)$
    \item $x = 3$
    \item $\frac{1}{2}\log_7(x) - 2\log_7(y) - \log_7(z)$
    \item $\log_4\left(\frac{x^2\sqrt[3]{y}}{z^4}\right)$
    \item $x = -\frac{1}{2}$
    \item $\frac{\log(7)}{\log(3)}$
    \item $\frac{\ln(12)}{\ln(5)}$
    \item $2$
    \item $2\log_{10}(x) + \frac{1}{2}\log_{10}(y) - 3\log_{10}(z)$
    \item $\log_6\left(\frac{xy^2}{\sqrt{z}}\right)$
    \item $x = 3$
\end{enumerate}

\subsection*{Medium Hard Problems (37-46)}
\begin{enumerate}[start=37]
    \item $2\log_3(x) + \frac{1}{3}\log_3(y) - 4\log_3(z+1)$
    \item $\log_7\left(\frac{x^{2/3}z^{3/2}}{y^{1/4}}\right)$
    \item $x = 5$
    \item $\frac{3}{4}$
    \item $3\log_4(x) + \frac{3}{4}\log_4(y) - \frac{3}{2}\log_4(z^2+1)$
    \item $x = 7$
    \item $\log_8(32) = \frac{5}{3}$
    \item Proof using change of base formula
    \item $x = 5$
    \item $2\log_{10}(x+1) + \frac{1}{3}\log_{10}(y-2) - \frac{5}{3}\log_{10}(z^2+4)$
\end{enumerate}

\subsection*{Hard Problems (47-52)}
\begin{enumerate}[start=47]
    \item $\log_6\left(\frac{6 \cdot 4}{3}\right) = \log_6(8)$
    \item $x = 8, y = 4$
    \item Proof using change of base formula
    \item $\frac{2}{3}\log_5(x) + \frac{3}{4}\log_5(y+1) + \frac{1}{5}\log_5(z-2) - \frac{4}{5}\log_5(w^2-1) - \frac{1}{6}\log_5(t+3)$
    \item $x = 5$
    \item $\log_2(45) = 2a + ab + 2$
\end{enumerate}

\subsection*{Extra Hard Problems (53-72)}
\begin{enumerate}[start=53]
    \item $x = 64$
    \item $\log_a(abc) = 1$ (when defined)
    \item $x = 2$
    \item $x = 64$
    \item Proof by change of base: LHS = $\log_a(x)\left(\frac{11}{6}\right)$ = RHS
    \item $x = 4$
    \item $x = 2 \pm \sqrt{6}$ (but check domain: only $x = 2 + \sqrt{6}$ is valid)
    \item $\frac{2}{9}$
    \item $x = 8, y = 8$
    \item Identity holds when $a, c > 0$, $a, c \neq 1$, and $b > 0$
    \item $x = 1$
    \item $1$
    \item $x = 3$
    \item $\log_a(n!)$
    \item $x = \frac{5}{3}$ (verify domain)
    \item $\frac{3p + 5q}{4}$
    \item $x = 4$ and $x = 8$
    \item Proof using change of base formula
    \item $f(x) = \log_2\left(\frac{2^x}{1+2^x}\right)$; solution involves symmetry properties
    \item $a = 2^k, b = 3^k, c = 4^k$ for some positive real $k$
\end{enumerate}



%%%%%%%%%%%%%%%%%%%%%%%%%%%%%%%%%%%%%%%%%%%%%%%%%%%%%%%%%%%%%%%%%%%%%%%%%%%%%%%%%%%%%%%%
%%%                                 S O L U T I O N S                                %%%
%%%%%%%%%%%%%%%%%%%%%%%%%%%%%%%%%%%%%%%%%%%%%%%%%%%%%%%%%%%%%%%%%%%%%%%%%%%%%%%%%%%%%%%%
\newpage
\chead{Answers}
\section*{Answers}






% \begin{tikzpicture}
%     % Define the matrix of nodes for the table
%     \matrix (sign_table) [
%         matrix of math nodes,
%         nodes={
%             text width=2cm, 
%             align=center, 
%             text height=1.5ex, 
%             text depth=1ex
%         },
%         column sep=0cm, 
%         row sep=0.5ex
%     ] {
%         % Row 1: Variable and number line with tick marks
%         x & -\infty & & -2 & & 3 & & +\infty \\
%         % Row 2: Sign of the first factor
%         x+2 & & - & |[fill=white,draw=black,circle,inner sep=1pt,label=below:$0$]| & + & + & & \\
%         % Row 3: Sign of the second factor
%         x-3 & & - & - & |[fill=white,draw=black,circle,inner sep=1pt,label=below:$0$]| & + & & \\
%         % Row 4: Sign of the entire function
%         f(x) & & + & |[label=below:$\phantom{0}$]| & - & |[label=below:$\phantom{0}$]| & + & \\
%     };

%     % Draw horizontal lines for the table rows
%     \foreach \row in {1,2,3} {
%         \draw (sign_table-\row-1.north west) -- (sign_table-\row-8.north east);
%     }
%     % Draw the final horizontal line
%     \draw (sign_table-4-1.south west) -- (sign_table-4-8.south east);
    
%     % Draw the number line arrows
%     \draw[latex-latex] (sign_table-1-2) -- (sign_table-1-8);
    
%     % Add vertical lines for critical points
%     \draw (sign_table-1-4.south) -- (sign_table-4-4.south);
%     \draw (sign_table-1-6.south) -- (sign_table-4-6.south);

%     % Label critical points on the number line
%     \node[below=1.5ex] at (sign_table-1-4) {$-2$};
%     \node[below=1.5ex] at (sign_table-1-6) {$3$};
    
%     % Style the zero nodes and place signs within the table
%     \draw (-2.8, -1.8) node {$+$};
%     \draw (0.8, -1.8) node {$+$};
%     \draw (-0.8, -2.5) node {$+$};
%     \draw (1.8, -2.5) node {$+$};
% \end{tikzpicture}

% \begin{tikzpicture}
% \matrix (m) [matrix of math nodes, 
%              column sep=0cm, row sep=0pt,
%      nodes={text width=20mm, align=center, 
%             text height=3ex, text depth=1.5ex}]
% {
% 3(x-2)^2    &   +   &   +   &   +   &\\
% (x+3)       &   -   &   +   &   +   &\\
% p(x)        &   -   &   +   &   +   &\\
% };
% \foreach \i in {1, 2, 3}
% {
% \draw  (m-\i-1.north west) -- (m-\i-4.north east);
% \draw  (m-1-\i.north east) -- (m-3-\i.south east);
% }
% \draw   (m-3-1.south west) -- (m-3-4.south east);
% %
% \node[above] at (m-1-2.north east) {$-3$};
% \node[above] at (m-1-3.north east) {$2$};
% \end{tikzpicture}







% \begin{tikzpicture}
% \tkzTabInit[lgt=3,espcl=2,deltacl=0]
%   { /.8, $3(x-2)^2$ /.8, $x+3$ /.8, $p(x)$ /.8}
%   {,$-3$,$2$,} % four main references
% \tkzTabLine {,+,t,+,z,+,} % seven denotations
% \tkzTabLine {,-,z,+,t,+,}
% \tkzTabLine {,-,z,+,z,+,}
% \end{tikzpicture}






% \begin{tikzpicture}[scale=.80]

% % variables and sign array
% %
% \def\h{.9};                         % row height 
% \def\w{2.5};                        % column width
% %
% \def\zeros{{"-2", "-1", 0, 1}};         % zeros and point of not existence
% \def\factors{{"x+2", "x", "x^2-1", "f(x)"}};            
% \def\signs{{{"-"    ,"\nexists" ,"+"    ,       ,"+"    ,       ,"+"    ,       ,"+"},          % "\nexists" is for point of not existence
%             {"-"    ,           ,"-"    ,       ,"-"    ,"0"    ,"+"    ,       ,"+"},
%             {"+"    ,           ,"+"    ,"0"    ,"-"    ,       ,"-"    ,"0"    ,"+"},
%             {"+"    ,"\nexists" ,"-"    ,"0"    ,"+"    ,"0"    ,"-"    ,"0"    ,"+"},
%             }};
% %
% \pgfmathsetmacro\nzeros{dim(\zeros)-1};         % number of zeros
% \pgfmathsetmacro\ncol{2*dim(\zeros)};           % Number of columns
% \pgfmathsetmacro\nrow{dim(\factors)-1};         % number of rows

% % Abscissa axis
% %
% \draw[->]   (-\w, 0)            -- ({(\nzeros+1)*\w+0.125}, 0)      node[above]{$x$};
% \draw       (-\w, {-\nrow*\h})  -- ({(\nzeros+1)*\w}, {-\nrow*\h});

% % Zeros and vertical (dotted) lines
% %
% \foreach \i in {0,...,\nzeros} {
%     \draw[dotted, gray!80]  (\i*\w, {-(\nrow+0.125)*\h})    -- (\i*\w, 0);
%     \draw                   (\i*\w, 0)                      -- (\i*\w, +0.125) node[above] {$\pgfmathparse{\zeros[\i]}\pgfmathresult$}; 
% }

% Signs and zeros table
%
% \foreach \row in {0,...,\nrow} {
%     \node[left]     at (-\w-0.125, {-(\row+0.5)*\h})        {\footnotesize signs of\,  $\pgfmathparse{\factors[\row]}\pgfmathresult$};
%     \foreach \col in {0,...,\ncol} {
%         \node       at ({(\col-1)*\w/2}, {-(\row+0.5)*\h})  {$\pgfmathparse{\signs[\row][\col]}\pgfmathresult$};    % <-- (1)
%     }                               
% }       
% \end{tikzpicture}

