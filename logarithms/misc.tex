\subsection*{Miscellaneous}
\chead{Miscellaneous}

\begin{enumerate}[resume]

\item Change of base: Express $\log_3(7)$ in terms of common logarithms

{\color{blue}
\begin{align*}
       \log_3(7)&= \dfrac{\log(7)}{\log(3)}\\
    &=\dfrac{0.8451}{0.4771}\\
    \Aboxed{&= 1.7712}
\end{align*}
}

    
\item Change of base: Express $\log_5(12)$ in terms of natural logarithms

{\color{blue}
\begin{align*}
       \log_5(12)&= \dfrac{\ln(12)}{\ln(5)}\\
    &=\dfrac{2.849}{1.6094}\\
    \Aboxed{&= 1.544}
\end{align*}

}
          


\item Solve for $x$: $2^{\log_2(x)} + 3^{\log_3(x)} = x + 1$
  {\color{blue}
  \begin{align*}
      2^{\log_2(x)} + 3^{\log_3(x)} &= x + 1\\
      x+x&=x+1\makebox[3cm]{\dotfill}\textit{Inverse Rule}\\
      2x&=x+1\\
      x&=1
  \end{align*}
  Check:
    \begin{align*}
      2^{\log_2(x)} + 3^{\log_3(x)} &\stackrel{?}{=} x + 1\\
      2^{\log_2(1)} + 3^{\log_3(1)} &\stackrel{?}{=} 1 + 1\\
      2^{0} + 3^{0} &\stackrel{?}{=} 1 + 1\\
      1+1&\stackrel{?}{=} 1 + 1\checkmark\makebox[3cm]{\dotfill}(2^0=3^0=1)
  \end{align*}

  \fbox{$x=1$}
% \begin{tcolorbox}[title=Solution, colback=blue!10!white, colframe=blue!75!black]
% $x=1$
% \end{tcolorbox}

}  
    
    
    
\item If $\log_a(x) = p$ and $\log_a(y) = q$, express $\log_{a^2}(\sqrt{x^3y^5})$ in terms of $p$ and $q$.
  {\color{blue}
  \begin{align*}
      \log_{a^2}(\sqrt{x^3y^5})&=\log_{a^2}\left(x^3y^5\right)^{1/2}\makebox[3cm]{\dotfill}\sqrt{a}=a^{1/2}\\
      &=\log_{a^2}\left(x^{3/2}y^{5/2}\right)\makebox[3cm]{\dotfill} (a\,b)^m=a^m\,b^m\\
      &=\log_{a^2}\left(x^{3/2}\right)+\log_{a^2}\left(y^{5/2}\right)\makebox[3cm]{\dotfill}\textit{Product Rule}\\
      &=\frac{\frac{3}{2}}{2}\log_{a}\left(x\right)+\frac{\frac{5}{2}}{2}\log_{a}\left(y\right)\makebox[3cm]{\dotfill} \log_{m^n}x^y=\frac{y}{n}\log_mx\\\\
      \Aboxed{&=\dfrac{3}{4}p+\dfrac{5}{4}q}
  \end{align*}
  
   }  
    
    \item Prove that: $\log_a(b) \cdot \log_b(c) \cdot \log_c(d) \cdot \log_d(a) = 1$
   
    {\color{blue}
    Use the following change of base:
    \begin{align*}
        \log_b(c)&=\dfrac{\log_a(c)}{\log_a(b)}\\
        \log_c(d)&=\dfrac{\log_a(d)}{\log_a(c)}\\
        \log_d(a)&=\dfrac{\log_a(a)}{\log_a(d)}=\dfrac{1}{\log_a(d)}
    \end{align*}
    Plug in:
    \begin{align*}
     \log_a(b) \cdot \dfrac{\log_a(c)}{\log_a(c)} \cdot \dfrac{\log_a(d)}{\log_a(c)} \cdot \dfrac{\log_a(a)}{\log_a(d)}=\cancel{\log_a(b)} \cdot \dfrac{\cancel{\cancel\log_a(c)}}{\cancel{\log_a(b)}} \cdot \dfrac{\cancel{\log_a(d)}}{\cancel{\log_a(c)}} \cdot \dfrac{1}{\cancel{\log_a(d)}}=1
    \end{align*}
    
    }
    
    \item Solve the functional equation: $f(x) + f(1-x) = 1$ where $f(x) = \log_2\left(\frac{2^x}{1+2^x}\right)$

    {\color{blue}
    \begin{align*}
        f(x)+f(1-x)&=\log_2\left(\frac{2^x}{1+2^x}\right)+\log_2\left(\frac{2^{(1-x)}}{1+2^{(1-x)}}\right)\\
        2&=\log_2\left(\frac{2^x}{1+2^x}\cdot\frac{2^{(1-x)}}{1+2^{(1-x)}}\right)\\
        &=\log_2\left(\dfrac{2^{x+1-x}}{\left(1+2^x\right)\left(1+2^{(1-x)}\right)}\right)\\
        &=\log_2\left(\dfrac{2}{\left(1+2^x\right)\left(1+2^{(1-x)}\right)}\right)\\
        2&=\left(1+2^x\right)\left(1+2^{(1-x)}\right)\\
        &=1+2^{1-x}+2^x+2\\
        0&=2^{x+1}+2^x+2\\
        0
    \end{align*}
    }
    
    \item Challenge: If $\log_2(a) + \log_3(b) + \log_4(c) = \log_6(abc)$, find the relationship between $a$, $b$, and $c$.

    \item Solve the system:
    
    \begin{align}
        \log_2(x) + \log_2(y) &= 5 \\
        \log_2(x) - \log_2(y) &= 1
    \end{align}
    
    {\color{blue}
    Add the two equations to eliminate $x$:
    \begin{align*}
        2\log_2(x)&=6\\
        \log_2(x)&=3\\
        x&=2^3=8
    \end{align*}

Plug in (2) to solve for $y$:
    \begin{align*}
        \log_2(x) - \log_2(y) &= 1\\
        \log_2(8) - \log_2(y) &= 1\\
        3-\log_2(y)&=1\\
        \log_2(y)&=2\\y&=2^2=4
    \end{align*}
    
    Check: Plug in $(8,4)$ in both equations  
    
\begin{align*}
        \log_2(8) + \log_2(4) &\stackrel{?}{=} 5 \\
        \log_2(8) - \log_2(4) &\stackrel{?}{=} 1\\
        &\\
        \log_2(2^3) + \log_2(2^2) &\stackrel{?}{=} 5 \\
        \log_2(2^3) - \log_2(2^2) &\stackrel{?}{=} 1\\ 
        &\\
        3+2&\stackrel{?}{=}5\\
        3-2&\stackrel{?}{=}1\\
        &\\
        5&\stackrel{?}{=}5\checkmark\\
        1&\stackrel{?}{=}1\checkmark\\
    \end{align*}
 
    \fbox{$(x,y)=(8,4)$}   }
%   \begin{tcolorbox}[title=Solution, colback=blue!10!white, colframe=blue!75!black]
% $(x,y)=(8,4)$
% \end{tcolorbox}  


    
   \item Solve the system:
    \begin{align}
        \log_2(xy) &= 6 \\
        \log_2(x^2y) &= 9
    \end{align}
    {\color{blue}
    Subtract the two equations
    \begin{align*}
        \log_2(x^2y)-\log_2(xy)&=9-6\\
        \log_2\left(\dfrac{x^2y}{xy}\right)&=3\makebox[5cm]{\dotfill}\textit{Quotient Property}\\
        \log_2(x)&=3\\
        x&=2^3\makebox[5cm]{\dotfill}\textit{Definition}\\
        x&=8
    \end{align*}
    Solve for $y$: Plug $x=8$ into the first equation:
    \begin{align*}
         \log_2(xy) &= 6\\
         \log_2(8y)&\stackrel{?}{=}6\\
         8y&=2^6 \makebox[5cm]{\dotfill} \textit{Definition}\\
         8y&=64\\
         y&=\frac{64}{8}=8
    \end{align*}

    Check: Plug $(8,8)$ in both equations
    \begin{align*}
        \log_2(xy) &= 6 \\
        \log_2(x^2y) &= 9
        &\\
        \log_2(8\times 8 &\stackrel{?}{=} 6 \\
        \log_2(8^2\times 8) &\stackrel{?}{=} 9\\
        &\\
        \log_2(2^3\times2^3) &\stackrel{?}{=} 6 \\
        \log_2((2^3)^2\times 2^3) &\stackrel{?}{=} 9\\
        &\\
        \log_2(2^{3+3}) &\stackrel{?}{=} 6 \\
        \log_2(2^6\times 2^3) &\stackrel{?}{=} 9\\     
        &\\
        \log_2(2^6) &\stackrel{?}{=} 6 \\
        \log_2(2^{6+3}) &\stackrel{?}{=} 9\\   
         &\\
        \log_2(2^6) &\stackrel{?}{=} 6 \\
        \log_2(2^9 &\stackrel{?}{=} 9\\     
        &\\
        6 &\stackrel{?}{=} 6 \checkmark\\
       9 &\stackrel{?}{=} 9\checkmark\\   
    \end{align*}

    
\fbox{$(x,y)=(8,8)$}
    
%   \begin{tcolorbox}[title=Solution, colback=blue!10!white, colframe=blue!75!black]
% $(x,y)=(8,8)$
% \end{tcolorbox} 
    
}% end color{blue}

    

    
    \end{enumerate}
    
  