\chead{Expand Completely}

\newpage
%%%%%%%%%%%%%%%%%%%%%%%%%%%%%%%%%%%%%%%%%%%%%%%%%%%%%%%%%%%%%%%%%%%%%%%%%%%%%%%
%%%
%%%                     EXPAND COMPLETELY
%%%
%%%%%%%%%%%%%%%%%%%%%%%%%%%%%%%%%%%%%%%%%%%%%%%%%%%%%%%%%%%%%%%%%%%%%%%%%%%%%%%
\noindent\subsection*{Solve Exponential Problems}
\begin{tcolorbox}[title=Note, colback=blue!10!white, colframe=blue!75!black]
Use the properties to write in simplest form. In particular, use the power, product and quotient rules. 
\end{tcolorbox}
\vspace{3.6cm}
\begin{enumerate}[resume]
    \item  $2^{x+1}=6^x$:
    {\color{blue}
    \begin{align*}
        2^{x+1}&=6^x\\
        \ln{2^{x+1}}&=\ln{6^x}\makebox[5cm]{\dotfill}\textit{Equality of Base}\\
        (x+1)\ln{2}&=x\ln{6}\makebox[5cm]{\dotfill}\textit{Power Rule}\\
        x\ln{2}+\ln{2}&=x\ln{6}\\
        x(\ln{2}-\ln{6})&=-\ln{2}\\
        x&=\dfrac{-\ln{2}}{\ln{2}-\ln{6}}\\
        \Aboxed{x&=\dfrac{\ln{2}}{\ln{6}-\ln{2}}}
    \end{align*}
    }


    
    \item  $5\cdot 2^{2x-1}=7^{x+1}$:
    {\color{blue}
        \begin{align*}
        5\cdot 2^{2x-1}&=7^{x+1}\\
        \ln{5\cdot 2^{2x-1}}&=\ln{7^{x+1}}\makebox[5cm]{\dotfill}\textit{Equality of Base}\\
        \ln{5}+\ln{2^{2x-1}}&=\ln{7^{x+1}}\makebox[5cm]{\dotfill}\textit{Power Rule}\\
        \ln{5}+(2x1)\ln{2}&=(x+1)\ln{7}\makebox[5cm]{\dotfill}\textit{Power Rule}\\
        \ln{5}+2x\ln{2}+\ln{2}&=x\ln{7}=\ln{7}\\
        2x\ln{2}-x\ln{7}&=\ln{7}-\ln{5}-\ln{2}\\
        x(2\ln{2}-\ln{7})&=\ln7-\ln{5}-\ln{2}\\
        \Aboxed{x&=\dfrac{\ln{7}-\ln{5}-\ln{2}}{2\ln{2}-\ln{7}}}        
    \end{align*}
    }



    \item  $3\cdot 5^{3x-1}=9^{x+2}$:
    
        {\color{blue}
        \begin{align*}
        3\cdot 5^{3x-1}&=9^{x+2}\\
        \ln{\left(3\cdot 5^{3x-1}\right)}&=\ln{\left(9^{x+2}\right)}\makebox[5cm]{\dotfill}\textit{Equality of Base}\\ 
        \ln{3}+\ln{5^{3x-1}}&=\ln{9^{x+2}}\makebox[5cm]{\dotfill}\textit{Product Rule}\\ 
        \ln{3}+(3x-1)\ln{5}&=(x+2)\ln{9}\makebox[5cm]{\dotfill}\textit{Power Rule}\\ 
        \ln{3}+3x\ln{5}-\ln{5}&=x\ln{9}+2\ln{9}\\
        3x\ln{5}-x\ln{3^2}&=\ln{5}-\ln{3}+2\ln{3^2}\\
        x(3\ln{5}-2\ln{3})&=\ln{5}-\ln{3}+4\ln{3}\makebox[5cm]{\dotfill}\textit{Power Rule}\\ 
        x(3\ln{5}-2\ln{3})&=\ln{5}+3\ln{3}\\
        \Aboxed{x&=\dfrac{\ln{5}+3\ln{3}}{{3\ln{5}-2\ln{3}}}}
    \end{align*}
    }
    
    
    OR
    {\color{teal}
    
        \begin{align*}
        3\cdot 5^{3x-1}&=9^{x+2}\\
        5^{3x-1}&=\dfrac{9^{x+2}}{3}\\
        \ln{5^{3x-1}}&=\ln{\dfrac{9^{(x+2)}}{3}}\makebox[5cm]{\dotfill}\textit{Equality of Base}\\
        \ln{5^{3x-1}}&=\ln{9^{x+2}}-\ln{3}\makebox[5cm]{\dotfill}\textit{Quotient Rule}\\ 
         (3x-1)\ln{5}&=(x+2)\ln{9}-\ln{3}\makebox[5cm]{\dotfill}\textit{Power Rule}\\ 
         3x\ln{5}-\ln{5}&=x\ln{9}+2\ln{9}-\ln{3}\\
         3x\ln{5}-\ln{5}&=x\ln{3^2}+2\ln{3^2}-\ln{3}\\
         3x\ln{5}-\ln{5}&=2x\ln{3}+4\ln{3}-\ln{3}\makebox[5cm]{\dotfill}\textit{Power Rule}\\
         3x\ln{5}-2x\ln{3}&=\ln{5}+3\ln{3}\\
         x(3\ln{5}-2\ln{3})&=\ln{5}+3\ln{3}\\
         \Aboxed{x&=\dfrac{\ln{5}+3\ln{3}}{3\ln{5}-2\ln{3}}}
    \end{align*}
    }
OR\\

    {\color{orange}
    
        \begin{align*}
        3\cdot 5^{3x-1}&=9^{x+2}\\
        5^{3x-1}&=\dfrac{\left(3^2\right)^{(x+2)}}{3}\\
        5^{3x-1}&=\dfrac{3^{2(x+2)}}{3}\makebox[5cm]{\dotfill}\textit{Power Rule}\\ 
        5^{3x-1}&=\dfrac{3^{2x+4}}{3}\\
        5^{3x-1}&=3^{2x+4-1}\\
        5^{3x-1}&=3^{2x+3}\\
        \ln 5^{3x-1}&=\ln 3^{2x+3}\makebox[5cm]{\dotfill}\textit{Equality of Base}\\
        (3x-1)\ln{5}&=(2x+3)\ln{3}\\
        3x\ln{5}-\ln{5}&=2x\ln{3}+3\ln{3}\\
        3x\ln{5}-2x\ln{3}=&\ln{5}+3\ln{3}\\
        x(3\ln{5}-2\ln{3})&=\ln{5}+3\ln{3}\\
        \Aboxed{x&=\dfrac{\ln{5}+3\ln{3}}{3\ln{5}-2\ln{3}}}
    \end{align*}
    }



    \item  $5\cdot 2^{2x-1}=7^{x+1}$:
    {\color{blue}
        \begin{align*}
        5\cdot 2^{2x-1}&=7^{x+1}\\
        \ln{5\cdot 2^{2x-1}}&=\ln{7^{x+1}}\makebox[5cm]{\dotfill}\textit{Equality of Base}\\
        \ln{5}+\ln{2^{2x-1}}&=\ln{7^{x+1}}\makebox[5cm]{\dotfill}\textit{Power Rule}\\
        \ln{5}+(2x1)\ln{2}&=(x+1)\ln{7}\makebox[5cm]{\dotfill}\textit{Power Rule}\\
        \ln{5}+2x\ln{2}+\ln{2}&=x\ln{7}=\ln{7}\\
        2x\ln{2}-x\ln{7}&=\ln{7}-\ln{5}-\ln{2}\\
        x(2\ln{2}-\ln{7})&=\ln7-\ln{5}-\ln{2}\\
        \Aboxed{x&=\dfrac{\ln{7}-\ln{5}-\ln{2}}{2\ln{2}-\ln{7}}}        
    \end{align*}
    }








    
    \item  $e^{2x}+4e^x=12$:
    {\color{blue}
        \begin{align*}
        e^{2x}+4e^x&=12\\
        \left(e^x\right)^2+4e^x7-12&=0\makebox[5cm]{\dotfill}\textit{Power Rule}\\
        u^2+4u-12&=0\makebox[5cm]{\dotfill}\text{let }u = e^x>0\\
        (u+6)(u-2)&=0\\
        u=-6&;u=2
    \end{align*}
Since $u=e^x$, which is always positive, we reject $u=-6$ as a solution and keep $u=2$ to solve for $x$. 

\begin{align*}
    u=e^x&=2\\
    \ln{e^x}&=\ln{2}\makebox[5cm]{\dotfill}\textit{Equality of Base}\\
    \Aboxed{x&=\ln{2}\makebox[5cm]{\dotfill}\textit{Power Rule}}
\end{align*}

Check:

\begin{align*}
    e^{2x}+4e^x&=12\\
    e^{2\ln{2}}+4e^{\ln{2}}&\stackrel{?}{=}12\\
    e^{\ln{2^2}}+4e^{\ln{2}}&\stackrel{?}{=}12\\
    2^2+4(2)&\stackrel{?}{=}12\\
    4+8&\stackrel{?}{=}12\checkmark
\end{align*}


    }

    
    \item  $\dfrac{e^x+e^{-x}}{2}=3$:
    {\color{blue}
        \begin{align*}
        \dfrac{e^x+e^{-x}}{2}&=3\\
        e^x+e^{-x}&=6\\
        e^x+\dfrac{1}{e^x}&=6\\
        \dfrac{e^{2x}+1}{e^x}&=6\\
        e^{2x}+1&=6e^x\\
        \left(e^{x}\right)^2-&6e^x+1=0\makebox[5cm]{\dotfill}\textit{Power Rule}
    \end{align*}

    
    Let $u=e^x$, then the equation becomes
    \[
        u^2-6x+1=0
    \]
Use the quadratic formula
\begin{align*}
    u&=\dfrac{-b\pm\sqrt{b^2-4ac}}{2a}\\
    &=\dfrac{-(-6)\pm\sqrt{(-6)^2-4(1)(1)}}{2(1)}\\
    &=\dfrac{6\pm\sqrt{36-4}}{2}\\
    &=\dfrac{6\pm\sqrt{32}}{2}\\
    &=\dfrac{6\pm 4\sqrt{2}}{2}\\
    &=3\pm 2\sqrt{2}
\end{align*}
Back to $x$: $u=e^x=3\pm2\sqrt{2}$. Both vales $3+2\sqrt{2}$ and $3-2\sqrt{2}$ are positive, so they are both solutions to the problem. 

\begin{align*}
    e^x&=3\pm 2\sqrt{2}\\
    \ln{e^x}&=\ln{3\pm 2\sqrt{2}}\makebox[5cm]{\dotfill}\textit{Equality of Base}\\
    \Aboxed{x&=\ln{3\pm 2\sqrt{2}}\makebox[5cm]{\dotfill}\textit{Inverse Rule}}
\end{align*}

}
    


\end{enumerate}